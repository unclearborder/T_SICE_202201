% interactnlmsample.tex
% v1.05 - August 2017

\documentclass[]{interact}

\usepackage{epstopdf}% To incorporate .eps illustrations using PDFLaTeX, etc.
\usepackage[caption=false]{subfig}% Support for small, `sub' figures and tables
%\usepackage[nolists,tablesfirst]{endfloat}% To `separate' figures and tables from text if required
%\usepackage[doublespacing]{setspace}% To produce a `double spaced' document if required
%\setlength\parindent{24pt}% To increase paragraph indentation when line spacing is doubled

\usepackage[numbers,sort&compress]{natbib}% Citation support using natbib.sty
\bibpunct[, ]{[}{]}{,}{n}{,}{,}% Citation support using natbib.sty
\renewcommand\bibfont{\fontsize{10}{12}\selectfont}% Bibliography support using natbib.sty
\makeatletter% @ becomes a letter
\def\NAT@def@citea{\def\@citea{\NAT@separator}}% Suppress spaces between citations using natbib.sty
\makeatother% @ becomes a symbol again

\theoremstyle{plain}% Theorem-like structures provided by amsthm.sty
\newtheorem{theorem}{Theorem}[section]
\newtheorem{lemma}[theorem]{Lemma}
\newtheorem{corollary}[theorem]{Corollary}
\newtheorem{proposition}[theorem]{Proposition}

\theoremstyle{definition}
\newtheorem{definition}[theorem]{Definition}
\newtheorem{example}[theorem]{Example}

\theoremstyle{remark}
\newtheorem{remark}{Remark}
\newtheorem{notation}{Notation}

\begin{document}

% \articletype{ARTICLE TEMPLATE}% Specify the article type or omit as appropriate

\title{aaa}

\author{
\name{Yuya Okada\textsuperscript{a}\thanks{CONTACT Yuya~Okada. Email: okada@hfg.sc.e.titech.ac.jp}, Hiroki Sugawara\textsuperscript{b}, 
    Hiroaki Soya\textsuperscript{b}, Takeshi Hatanaka\textsuperscript{a}}
\affil{\textsuperscript{a}Tokyo Institute of Technology, Tokyo, Japan}
\affil{\textsuperscript{b}Yokogawa Electric Corporation, Tokyo, Japan}
}

\maketitle

\begin{abstract}

\end{abstract}

\begin{keywords}
% Sections; lists; figures; tables; mathematics; fonts; references; appendices
\end{keywords}


\section{Introduction}

\section{Problem Statement}
% Let us consider there are $M$ types heterogeneous robots, of which $r_m,~m\in\{1,...,M\}$ has $N_m$ robots each.
In this section, when assigning tasks to robots, we use performed tasks, the robots, times and targets where tasks reside as indices.
The following subsections describe each problem sets.

\subsection{Tasks index}


\subsection{Robot types and number}
Let us consider M types heterogeneous robots and the number $N_m$ of robot $m\in\{1,2,...M\}$ among them.
Where the set of robots $m$ is $\mathcal{R}_m=\{r^m_1,r^m_2,...,r^m_{N_m}\}$ and the set of all robots are $\mathcal{R}=\mathcal{R}_1\cup\mathcal{R}_2\cup ... \cup\mathcal{R}_M$.
In the paper A, heterogeneous robots were given the same index, but in this paper, each robot $m$ is given a different index becaouse of the different tasks it can perform.
The set of tasks that the robot $r^m_{i_m}$ can perform is 

\end{document}
