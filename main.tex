% interactnlmsample.tex
% v1.05 - August 2017

\documentclass[]{interact}

\usepackage{epstopdf}% To incorporate .eps illustrations using PDFLaTeX, etc.
\usepackage[caption=false]{subfig}% Support for small, `sub' figures and tables
%\usepackage[nolists,tablesfirst]{endfloat}% To `separate' figures and tables from text if required
%\usepackage[doublespacing]{setspace}% To produce a `double spaced' document if required
%\setlength\parindent{24pt}% To increase paragraph indentation when line spacing is doubled

\usepackage[numbers,sort&compress]{natbib}% Citation support using natbib.sty
\bibpunct[, ]{[}{]}{,}{n}{,}{,}% Citation support using natbib.sty
\renewcommand\bibfont{\fontsize{10}{12}\selectfont}% Bibliography support using natbib.sty
\makeatletter% @ becomes a letter
\def\NAT@def@citea{\def\@citea{\NAT@separator}}% Suppress spaces between citations using natbib.sty
\makeatother% @ becomes a symbol again

\theoremstyle{plain}% Theorem-like structures provided by amsthm.sty
\newtheorem{theorem}{Theorem}[section]
\newtheorem{lemma}[theorem]{Lemma}
\newtheorem{corollary}[theorem]{Corollary}
\newtheorem{proposition}[theorem]{Proposition}

\theoremstyle{definition}
\newtheorem{definition}[theorem]{Definition}
\newtheorem{example}[theorem]{Example}

\theoremstyle{remark}
\newtheorem{remark}{Remark}
\newtheorem{notation}{Notation}

\begin{document}

% \articletype{ARTICLE TEMPLATE}% Specify the article type or omit as appropriate

\title{aaa}

\author{
\name{Yuya Okada\textsuperscript{a}\thanks{CONTACT Yuya~Okada. Email: okada@hfg.sc.e.titech.ac.jp}, Hiroki Sugawara\textsuperscript{b}, 
    Hiroaki Soya\textsuperscript{b}, Takeshi Hatanaka\textsuperscript{a}}
\affil{\textsuperscript{a}Tokyo Institute of Technology, Tokyo, Japan}
\affil{\textsuperscript{b}Yokogawa Electric Corporation, Tokyo, Japan}
}

\maketitle

\begin{abstract}

\end{abstract}

\begin{keywords}
% Sections; lists; figures; tables; mathematics; fonts; references; appendices
\end{keywords}


\section{Introduction}

\section{Problem Statement}
% Let us consider there are $M$ types heterogeneous robots, of which $r_m,~m\in\{1,...,M\}$ has $N_m$ robots each.
% In this section, when assigning tasks to robots, we use performed tasks, the robots, times and targets where tasks reside as indices.
% Let us consider M types heterogeneous robots and the number $N_m$ of robot $m\in\{1,2,...M\}$ among them.
% Where the set of robots $m$ is $\mathcal{R}_m=\{r^m_1,r^m_2,...,r^m_{N_m}\}$ and the set of all robots are $\mathcal{R}=\mathcal{R}_1\cup\mathcal{R}_2\cup ... \cup\mathcal{R}_M$.
% In the paper A, heterogeneous robots were given the same index, but in this paper, each robot $m$ is given a different index becaouse of the different tasks it can perform.
% The set of tasks that the robot $r^m_{i_m}$ can perform is 

In this paper, we consider a problem in which tasks are scattered in a workspace $\mathcal{D}$ and a heterogeneous robot is used to execute the tasks by a predetermined time.
Therefore, task, robot, and time indices are needed to assign tasks to the robot.
In addition, the robot is expected to move to the location where the task exists in order to perform the task. 
Therefore, an index of the location where the task exists is also required. These locations are called targets.
This section describes the task, target, robot, and time indices and sets necessary for the problem addressed in this paper.

\subsection{Index and Sets for the Problem}
% Let consider a problem in which multiple tasks of $J$ types are scattered in a workspace $\mathcal{D}$, and $M$ types of robots are used to perform the tasks by a predetermined time.
Let consider the set of $J$ tasks $\mathcal{T}=\{\tau_1,\tau_2,...,\tau_J\}$. 
There are no dependencies among tasks, and the order of performing is not considered. 
The tasks are scattered in a workspace $\mathcal{D}$, and the locations where these tasks exist are called targets. 
A target is associated with some neighboring tasks, and the number of targets is $H$. 
The set of targets is $\mathcal{O} = \{o_1,o_2,...,o_H\}$ and the set of tasks that exist in target $h\in\{1,2,...,H\}$ is $\mathcal{T}_h\subset \mathcal{T}$. 
Although each target has a path to its neighbors, we do not consider this.
Next, consider $M$ types of heterogeneous robots and the number of robot $m\in\{1,2,...,M\}$ is $N_m\in\mathbb{N}$. 
The set of robot $m$ can be represented as $\mathcal{R}_m\in\{r^m_1,r^m_2,...,r^m_{N_m}\}$ and the set of robots as a whole as $\mathcal{R}=\mathcal{R}_1\cup ...\cup \mathcal{R}_M$. 
Each of the $M$ robots perform different tasks, and the set of tasks performed by robot $m$ is $\mathcal{T}_m\subset\mathcal{T}$. 
Finally, for time, we divide the entire time from the start time to the end time of all tasks into $L$ steps. 
The set of times is described by $\mathcal{L}=\{1,2,...,\}$.
For simplicity, all tasks are assumed to be performed during this one time step $l\in\mathcal{L}$.
Each task for each target be performed at least one period in the entire time, and the set of $k$th performed period set is $\mathcal{L}^k_{hj}\subset\mathcal{L}$. 
Let $l^{ks}_{hj} = \mathrm{arg}\min_{\mathcal{L}^k_{hj}} l$ be the start time and $l^{ke}_{hj} = \mathrm{arg}\max_{\mathcal{L}^k_{hj}} l$ be the end time, then all tasks must be performed at time $l$ satisfying $l^{ks}_{hj} \le l \le l^{ke}_{hj}$.
In this paper, we consider allocating robot $m$ at time $l$ satisfying $l^{ks}_{hj} \le l \le l^{ke}_{hj}$ to a target where $\mathcal{J}_{hm}=\mathcal{J}_{m}\cap\mathcal{J}_{h}$, the common set of $\mathcal{J}_h$ and $\mathcal{J}_m$, is $\mathcal{J}_{hm}\neq \varnothing$.

\subsection{Allocation Variables}
The allocation, i.e., the assignment of robots to targets and tasks, is described by a set of Boolen variables $\delta_{ijhl}$ that are set to 1 when the robot $r_i$ performs task $\tau_j$ on target $o_h$ at time $l$ (otherwise it is set to 0).
For convenience, $x$ is defined as the aggregated vector $x=[\delta_{ijhl}]_{i\in\mathcal{R},j\in\mathcal{T},h\in\mathcal{O},l\in\mathcal{L}}$.

\section{Mix-Integer Linear Programming}




\end{document}
